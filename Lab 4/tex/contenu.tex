%!TEX root = ../rapport.tex
%!TEX encoding = UTF-8 Unicode

% Chapitres "Introduction"

% modifié par Francis Valois, Université Laval
% 31/01/2011 - version 1.0 - Création du document


\label{s:experimentation}
\chapter*{Question 1}
\section*{a)}

On cherche à identifier $\beta$, pour ce faire, nous avons l'équation suivante:
\begin{equation}
\beta = \frac{D_n N_{de} X_e}{D_p N_{ab}W}
\end{equation}
Où
\begin{itemize}
\item $\beta$ est le gain en courant du transistor;
\item $D_n$ est le coefficient de diffusion des électrons;
\item $D_p$ est le coefficient de diffusion des trous;
\item $N_{de}$ est la densité de donneurs dans l'émetteur;
\item $N_{ab}$ est la densité d'accepteurs dans la base;
\item $X_e$ est l'épaisseur de l'émetteur;
\item W est l'épaisseur de la base.
\end{itemize}

On a alors que:
\begin{equation}
\beta = \frac{37.5\cdot 1.7\times 10^{19}\cdot 0.75\times 10^{-4}}{13 \cdot 1.7\times 10^{17} \cdot 0.8\times 10^{-4}} =270.433
\end{equation}

\section*{b)}
Le rapport $\frac{N_{de}}{N_{ab}}$ apparaît, car ils sont directement reliés à la hauteur de la  barrière de potentiel que les électrons doivent traverser. Si l’on augmente le ratio, les électrons doivent franchir une barrière plus haute et le courant résultant, selon la figure présentée à la page 44 des notes de cours, sera nécessairement plus grand. Le rapport $\frac{D_n}{D_p}$ apparaît, car c'est en partie la mobilité des électrons qui permet un fort gain en courant dans le transistor. Si cette mobilité est trop proche de celle des trous, le gain résultant sera plus faible. Pour que l'effet transistor existe, il faut un courant d'électrons supérieur au courant de trous.

\section*{c)}
La densité de courant d'électrons diffusant dans la base venant de l'émetteur est donnée par:
\begin{equation}
\frac{I_b}{S} = \frac{q D_n n_{b0} \mbox{exp}\left(\frac{e V_{be}}{k_B T}\right)}{W}\left[\frac{A}{cm^2}\right]
\end{equation}

En divisant par la charge élémentaire et en multipliant par 1s, on obtient la densité d'électrons diffusant dans la base venant de l'émetteur (N):
\begin{equation}
N = \frac{D_n n_{b0} \mbox{exp}\left(\frac{e V_{be}}{k_B T}\right)}{W} \left[\frac{electrons}{cm^2}\right]
\end{equation}

On insère les chiffres dans l'équation:

\begin{equation}
N = \frac{37.5 \cdot \frac{10^{20}}{1.7\times10^{17}}\cdot \mbox{exp}\left(\frac{0.6}{0.026}\right)}{0.8\times10^{-4}} = 2.9018 \times 10^{19} \left[\frac{electron}{cm^2}\right]
\end{equation}

\chapter*{Question 2}
Pour permettre une bonne diffusion de notre nouvelle application, une marque de commerce ainsi qu'un logo associé ont été créés. La création du logo s'est effectuée en deux étapes. La première a été de trouver la meilleure façon de représenter l'utilisation de satellites dans notre produit ainsi que la transmission sans fil. Dans notre cas, une planète avec un satellite en orbite autour de celle-ci semblait être un choix judicieux. La deuxième partie du logo devrait illustrer le principal but de notre projet qui est de distribuer de la puissance. Ainsi, un éclair est tout à fait approprié. En effectuant la symbiose des deux parties, nous obtenons le logo présenté à la figure \ref{img:1}.

Par la suite, un bon produit rime très souvent avec un bon nom. Ainsi, nous avons décidé de trouver un nom accrocheur qui explique directement le but de notre produit. L'acronyme "WLP" qui signifie "Wireless Laser Power" est un bon choix, car il indique directement que le produit sert à transférer de la puissance de façon sans-fil à l'aide d'un Laser. De plus, l'acronyme est simple à retenir et possède une sonorité agréable pour l'oreille.

\begin{figure}[htbp]
\centering
\includegraphics[scale=0.45]{fig/logo.png}
\caption{Logo}
\label{img:1}
\end{figure}