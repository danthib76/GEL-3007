%!TEX encoding = UTF-8 Unicode
%!TEX root = ../rapport.tex
% Chapitres "Conclusion"

% modifié par Francis Valois, Université Laval
% 31/01/2011 - version 1.0 - Création du document

\chapter{Conclusion}
\label{s:conclusion}

Au cours de ce septième laboratoire, nous nous sommes familiarisé avec l'implantation pratique d'un oscillateur triangulaire pouvant être employé afin de moduler un signal quelconque. Par ailleurs, nous nous sommes aussi familiarisé avec l'implantation d'un comparateur à hystérésis capable d'effectuer ladite modulation en procurant une protection additionnelle contre le buit. Aussi, nous avons su implanté de manière pratique un filtre Butterworth d'ordre 2 ayant pour fonction d'effectuer la démodulation du signal. De par la précision la similitude entre l'onde de sortie et l'onde d'entrée, nous avons pu constater l'efficacité de notre montage et les qualités de la réponse en fréquence du filtre Butterworth. Il est intéressant de noter le déphasage des signaux de sortie, déphasage tout de même significatif, qui provient de la fonction de filtrage qui induit un déphasage non nul dans sa fonction de transfert.