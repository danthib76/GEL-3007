%!TEX root = ../rapport.tex
%!TEX encoding = UTF-8 Unicode

% Chapitres "Introduction"

% modifié par Francis Valois, Université Laval
% 31/01/2011 - version 1.0 - Création du document


\label{s:experimentation}
\chapter*{Question 1}
\section*{a)}
Pour répondre à cette question, on utilise l'équation suivante:

\begin{equation}
V_{Dsat} + V_{bi} - V_G = V_p
\end{equation}

Où
\begin{itemize}
\item $V_{Dsat}$ est la tension de drain de saturation;
\item $V_{bi}$ est la barrière de potentiel;
\item $V_G$ est la tension à la grille;
\item $V_p$ est une constante qui dépend de la configuration;
\end{itemize}

Si
\begin{equation}
-V_p + V_{bi} + V_D \geq V_G
\end{equation}
On a alors que le transistor ne conduit pas. On détermine $V_p$ au moyen de l'équation suivante:
\begin{equation}
V_p =\frac{qN_d h^2}{2\epsilon}
\end{equation}
Où
\begin{itemize}
\item q est la charge élémentaire;
\item $V_{bi}$ est la barrière de potentiel;
\item $N_d$ est la densité effective d'états des électrons;
\item h est la hauteur du canal;
\item $\epsilon$ est la permittivité relative;
\end{itemize}
\begin{equation}
V_p = \frac{q\cdot 0.9\times 10^{23} \cdot \left(0.3\times 10^{-6}\right)^2}{2\cdot 12.9 \cdot \epsilon_0} = 5.681V
\end{equation}

Avec les données du problème, on trouve que $V_p = 5.681 V$. Il faut donc une tension drain de 1.88 V pour que le transistor se bloque.

\subsection*{b)}
En utilisant les développements de la question 1a) et en remplaçant $V_d$ par $V_{Dsat}$ on a qu'il faut une tension $V_{Dsat} = 2.88V$

\subsection*{c)}
On a que le courant de saturation $I_{Dsat}$ est donné par l'équation suivante:

\begin{equation}
I_{Dsat} = g_0 \left \lbrace V_{Dsat} - \frac{2\left[V_p^{3/2} - (V_{bi}-V_G)^{3/2}\right]}{2V_p^{3/2}}\right\rbrace
\end{equation}

La constante $g_0$ représente la conductance et est donnée par:
\begin{equation}
g_0 = \frac{e \mu_n N_d Zh}{L}
\end{equation}
Où
\begin{itemize}
\item $e$ est la charge élémentaire;
\item $\mu_n$ est la mobilité des électrons;
\item Z est la largeur du canal de conduction;
\item L est la longueur du canal;
\end{itemize}

En insérant les valeurs numériques du problème, on obtient: 
\begin{align}
I_{Dsat} = &\frac{e\cdot 0.8 \cdot 0.9\times 10^{23} \cdot 150\times 10^{-6} \cdot 0.3\times 10^{-6}}{1.1\times 10^{-6}}\cdot\\ 
&\left\lbrace 2.88 - \frac{2\left[5.68^{3/2} - (0.8+2)^{3/2}\right]}{3\cdot 5.68^{1/2}}\right\rbrace\\
&= 0.1906A
\end{align}
\subsection*{d)}

Une longueur L plus faible améliore la conductance $g_0$ du transistor et permet ainsi d'obtenir un courant identique avec une tension plus faible.

\chapter*{Question 2}
Notre application peut se séparer en deux utilisations distinctes, soit civile ou militaire. Dans le cas civil, il semble intéressant pour une compagnie d'énergie de vouloir utiliser notre méthode de transfert d'énergie pour alimenter des régions éloignées très complexes à desservir par des lignes de transmission classiques et ce, à moindre coût. Nous pouvons penser à Hydro-Québec qui pourrait vouloir alimenter des petites îles dans le nord ou une zone complètement située dans la région arctique du Québec. Un investisseur d'une compagnie d'énergie mondiale qui voudrait desservir les petites îles des différents océans pourrait, lui aussi, y trouver son compte facilement. De plus, utiliser ce type d'énergie permet aux compagnies d'énergie qui utilise beaucoup d'énergie fossile d'être meilleure pour l'environnement et d'avoir une meilleure image si elles optent pour notre système d'énergie à la place de petites usines qui produisent beaucoup de polluant. 

\paragraph{} Si notre produit permet aux producteurs d'énergie de déporter de la puissance vers d'autres lieux, il est tout aussi adapté au transfert d'énergie pour les gros consommateurs. Il est de plus en plus courant dans le domaine technologique ("datacenter", usine de semiconducteurs) de produire sa propre énergie aux alentours des bâtiments l'utilisant. Cette production requiert souvent une quantité considérable d'espace et notre technologie permettrait à ces compagnies de déporter la production dans des zones plus éloignées et souvent moins couteuses sachant que les terrains utilisés pour les datacenter et autres sont souvent stratégiques et limités.

\paragraph{} Du côté militaire, les utilisations peuvent être diverses. La principale utilisation consiste au transfert d'énergie dans des zones reculées pour alimenter l'armement nécessitant beaucoup de puissance. Notre système pourrait même être modifié pour devenir lui même une arme de défense ou d'attaque en modifiant les paramètres des Lasers et en utilisant un système de guidage plus réactif au niveau du satellite ce qui permettrait un champ d'action aussi gros que la planète. Pour un gouvernement comme celui des États-Unis qui investit beaucoup dans le domaine militaire, notre produit est vraiment bien adapté. 

\paragraph{} Ainsi, que les investisseurs proviennent du domaine militaire ou du domaine civil, tous ceux qui désirent un système de transfert d'énergie efficace et à moindre coût devraient aider à la production réelle du produit.
