%!TEX root = ../rapport.tex
%!TEX encoding = UTF-8 Unicode

% Chapitres "Introduction"

% modifié par Francis Valois, Université Laval
% 31/01/2011 - version 1.0 - Création du document


\label{s:experimentation}
\chapter{Question 1}
\section{1)}

On cherche à calculer la longueur d'onde d'un électron qui possède une énergie de 100keV. On utilise premièrement la relation suivante:
\begin{equation}
E = \frac{1}{2} m_e v^2
\end{equation}
\begin{itemize}
\item Où $m_e$ est la masse de l'électron en kg
\item Où E est son énergie en Joules
\item Où v est sa vitesse en m/s
\end{itemize}

On a donc que:
\begin{equation}
v = \sqrt{\frac{2E}{m}} = \sqrt{\frac{2 \cdot 100\times 10^{3} \cdot 1.602\times 10^{-19}}{0.9109\times 10^{-30}}} = 187.55373\times 10^{6} \left[\frac{m}{s}\right]
\end{equation}

On trouve par la suite le "momentum" de l'électron au moyen de la relation suivante:
\begin{equation}
p = m_e v = 0.9109\times 10^{-30} \cdot 187.55373\times 10^{6} = 170.85\times 10^{-24} \left[\frac{kg \cdot m}{s}\right]
\end{equation}
\begin{itemize}
\item Où p est le "momentum" de l'électron en $\left[\frac{kg \cdot m}{s}\right]$
\end{itemize}

On utilise ensuite la relation de De Broglie afin de trouver la longueur d'onde de l'électron à 100keV:
\begin{equation}
\lambda = \frac{h}{p} = 3.8783 \times 10^{-12} [m] = 3.8783 [pm]
\end{equation}
\begin{itemize}
\item Où $\lambda$ est la longueur d'onde en mètres
\item Où h est la constante de Planck en $\left[\frac{m^2 \cdot kg}{s}\right]$
\end{itemize}
\section{2)}
Comme la particule est un photon, on emploi l'équation suivante afin de trouver le "momentum":
\begin{equation}
p = \frac{E}{c} = \frac{7 \cdot 1.602\times 10^{-19}}{c} = 3.741\times 10^{-27} \left[\frac{kg \cdot m}{s}\right]
\end{equation}
\begin{itemize}
\item Où c est la vitesse de la lumière en $\left[\frac{m}{s}\right]$
\end{itemize}

On utilise par la suite la même relation qu'à la question 1 pour trouver la longueur d'onde
\begin{equation}
\lambda = \frac{h}{p} = \frac{h}{3.741\times 10^{-27}} = 177.12\times 10^{-9} = 177.12nm
\end{equation}
\section{3)}
On sait que la puissance du faisceau est de 2GW. On a aussi que la durée du faisceau est de 1 s. On connaît d'ailleurs l'énergie des électrons. On utilise donc la puissance du laser afin de déterminer le nombre d'électrons et on effectue la somme des momentums de chacun. Par la suite, il suffit de supposer que le momentum des électrons varie en 1 seconde afin de déterminer l'accélération et ainsi la force.

\begin{equation}
P_{e} = \frac{E_{e}}{t} = \frac{100 \times 10^3 \left[keV\right]}{1 \left[s\right]} = \frac{100 \times 10^3 \cdot 1.602\times 10^{-19} \left[\frac{J}{eV}\right]}{1 \left[s\right]} = 1.602\times 10^{-14}\left[\frac{W}{électron}\right]
\end{equation}
\begin{itemize}
\item $P_{e}$ est la puissance d'un électron en [W]
\item $E_{e}$ est l'énergie d'un électron en [J]
\item t est le temps
\end{itemize}
On a par la suite que:
\begin{equation}
N = \frac{P_t}{P_e} = \frac{2 \times 10^{9}}{1.602\times 10^{-14}} = 1.2484\times 10^{23}\left[électron\right]
\end{equation}

\begin{itemize}
\item Où N est le nombre d'électrons
\item Où $P_t$ est la puissance totale du faisceau en [W]
\end{itemize}

Des développements faits au numéro 1, on a que
\begin{equation}
p = 170.85\times 10^{-24}\left[\frac{kg \cdot m}{s}\right]
\end{equation}

On peut trouver le momentum de l'ensemble des électrons du faisceau:
\begin{equation}
p_{T} = \sum_{N} p = Np = 1.2484\times 10^{23} \cdot 170.85\times 10^{-24} = 21.33\left[\frac{kg \cdot m}{s}\right]
\end{equation}
\begin{itemize}
\item Où $p_{T}$ est le momentum total
\end{itemize}

On trouve ainsi $F_e$
\begin{equation}
F_{e} = \frac{21.33 \left[\frac{kg \cdot m}{s}\right]}{1 s} = 21.33\left[N\right]
\end{equation}
\begin{itemize}
\item Où $F_{e}$ est la force de l'ensemble des électrons du faisceau [N]
\end{itemize}


On cherche maintenant la force de l'ensemble des photons du faiseau. On procède avec un développement analogue au développement précédent.

\begin{equation}
P_{c} = \frac{E_{c}}{t} = \frac{7 \left[eV\right]}{1 \left[s\right]} = \frac{7 \cdot 1.602\times 10^{-19} \left[\frac{J}{eV}\right]}{1 \left[s\right]} = 1.1214\times 10^{-18}\left[\frac{W}{électron}\right]
\end{equation}
\begin{itemize}
\item $P_{c}$ est la puissance d'un photon en [W]
\item $E_{c}$ est l'énergie d'un photon en [J]
\end{itemize}
On a par la suite que:
\begin{equation}
N_c = \frac{P_t}{P_c} = \frac{2 \times 10^{9}}{1.1214\times 10^{-18}} = 1.7835\times 10^{27}\left[photons\right]
\end{equation}

\begin{itemize}
\item Où $N_c$ est le nombre de photons
\end{itemize}

Des développements faits au numéro 2, on a que
\begin{equation}
p_c = 3.741\times 10^{-27} \left[\frac{kg \cdot m}{s}\right]
\end{equation}
\begin{itemize}
\item Où $p_c$ est le momentum des photons $\left[\frac{kg \cdot m}{s}\right]$
\end{itemize}

On peut trouver le momentum de l'ensemble des photons du faisceau:
\begin{equation}
p_{T} = \sum_{N_c} p_c = N_cp_c = 1.7835\times 10^{27} \cdot 3.741\times 10^{-27} = 6.67\left[\frac{kg \cdot m}{s}\right]
\end{equation}


On trouve ainsi $F_c$
\begin{equation}
F_{c} = \frac{6.67 \left[\frac{kg \cdot m}{s}\right]}{1 s} = 6.67\left[N\right]
\end{equation}
\begin{itemize}
\item Où $F_{c}$ est la force de l'ensemble des photons du faisceau [N]
\end{itemize}

\chapter{Question 2}
\section{Électronique à commande optique}
L'idée consiste à développer un dispositif de commutation électronique qui est commandé par un faiseau optique, sans l'apport d'énergie électrique afin de réaliser des dispositifs électroniques avec un rendement qui permettrait la portabilité de systèmes plus gourmands en énergie. L'idée est jointe au développement d'un accumulateur "optique" qui pourrait, au moyen du principe de réflexion totale interne ou au moyen d'une réfraction bien calculée, utiliser l'énergie sous forme lumineuse afin de commuter ou d'amplifier des signaux. Optimalement, on peut réénergiser les photons au moyen des champs électriques "parasites" localisés près des composantes lorsqu'ils ont été employés pour commuter ou amplifier. De ce fait, la réutilisation d'une portion de l'énergie permettrait d'obtenir des rendement de loin supérieurs aux rendement existants.
\section{Système d'accumulateur recharché au laser pour automobiles}
Le projet consiste à développer une méthode selon laquelle il serait possible d'utiliser une automobile avec un moteur électrique qui dispose d'un système d'alimentation à batterie rechargée au laser (par un faisceau à distance). Le génie derrière la technologie serait de développer un système qui permettrait d'énergiser des automobiles en mouvement de manière à assurer un mouvement continuel, sans besoin d'arrêt à de points fixes. Idéalement, les lieux de générations seraient alimentés par des sources autonomnes non reliées. 

\section{Système de transfert d'énergie sans-fil par laser}
Le but principal du projet est de permettre un transfert d'énergie de haute puissance sur de grandes distances. L'ordre de grandeur de la distance entre l'émetteur et le récepteur pourrait aller de quelques mètres à plusieurs milliers de kilomètres. Ainsi, il serait possible d'alimenter les dispositifs possédant des contraintes de poids et de volume plus efficacement en déportant le stockage de l'énergie. Dans la même optique, il serait possible de capturer l'énergie du soleil directement dans l'espace et la diffuser sur la terre de façon plus efficace que d'utiliser les panneaux solaires à mauvais rendement sur la terre. De plus, cela permettrait d'envoyer de l'énergie à des endroits où il est difficile d'en produire pour le moment.

\section{La radiation comme médium d'oeuvres d'art}
Dans ce projet, il est possible de se servir de la différence de puissances entre les différents électrons qui sont relâchés lors de la désintégration d'un objet radioactif pour dessiner une oeuvre d'art. Comme cette désintégration est imprévisible, il est possible d'obtenir une oeuvre d'art unique et hors du commun. Ainsi, à l'aide d'un détecteur d'électron, il serait possible d'obtenir des couleurs, des positions et des intensités sur un écran quelconque. Ce genre d'art abstrait pourrait même rivaliser avec les plus grands artistes.