%!TEX root = ../rapport.tex
%!TEX encoding = UTF-8 Unicode

% Chapitres "Introduction"

% modifié par Francis Valois, Université Laval
% 31/01/2011 - version 1.0 - Création du document


\label{s:experimentation}
\chapter{Question 1}
\section{a)}

On cherche à calculer la position du niveau de Fermi pour les trous et pour les électrons. La quantité $k_B T$ est égale a 26meV dans des conditions normales. Pour le GaAs, on a que les paramètres $N_c$ et $N_v$ des équations \ref{eq1} et \ref{eq2} sont respectivement égal à $4.7\times 10^{23}/m^3$ et à $7.0\times 10^{24}/m^3$. Les quantités n et p sont toutes deux égales à $9.0\times 10^{24}\left[\frac{electrons}{m^3}\right]$. 
\begin{align}
\label{eq1}
E_{Fn} - E_c &= k_B T\left(\ln\left(\frac{n}{N_c}\right) + 0.353 \frac{n}{N_c}\right)\\
\label{eq2}
E_{v} - E_{Fp} &= k_B T\left(\ln\left(\frac{p}{N_v}\right) + 0.353 \frac{p}{N_v}\right)
\end{align}
\begin{itemize}
\item Où $E_{Fn}$ est le niveau de Fermi des électrons en [J]
\item Où $E_{Fp}$ est le niveau de Fermi des trous en [J]
\item Où $N_c$ est la densité effective d'états des électrons en $\left[\frac{états}{m^3}\right]$
\item Où $N_v$ est la densité effective d'états des trous en $\left[\frac{états}{m^3}\right]$
\end{itemize}

On a alors que le niveau de Fermi pour les électrons est donné par:
\begin{equation}
E_{Fn} - E_c  = 26meV \left( 2.9522 + 6.76\right) = 252.5meV
\end{equation}

Et le niveau de Fermi pour les trous est donné par:

\begin{equation}
E_{v} - E_{Fp} = 26meV \left( 0.2513 + 0.4539 \right) =18.3meV
\end{equation}

\section{b)}
En termes d'énergie, on a que selon la condition de Bernard-Duraffourg $E_{Fn} - E_{Fp} \geq E_c - E_v \geq E_g \geq hf$ Dans le cas étudié, on a que $E_{Fn} - E_{Fp} = E_g + 252.5meV + 18.3meV = 1.6945eV$. On sait donc que le GaAs pourra laser de 1.424eV à 1.6945eV, soit une plage de longueur d'onde allant de 732.192[nm] à 870.675[nm]

\section{c)}
On calcule premièrement l'énergie fournie par le laser qui sert à pomper la lame d'GaAs:

\begin{equation}
E = \frac{c h}{\lambda} = \frac{c h}{600 \times 10^{-9}}
\end{equation}


\chapter{Question 2}
\textit{L'idée qui sera développée est celle du transfert sans fil d'énergie par laser.}

\section{Distribution résidentielle} Vu le nombre important de clients résidentiels vivant dans des régions rurales ou éloignées, il serait intéressant de développer l’alimentation sans fil sur longue portée afin de les desservir.
Pour ce faire, il est primordial de considérer l’apport d’une source de laser à bon rendement, capable d’émettre sur une longue portée avec la fréquence la plus basse possible afin de limiter les pertes de propagation. Si la fréquence est différente de la fréquence usuelle de 60Hz, il faudra impérativement des convertisseurs AC/DC et DC/AC de manière à pouvoir fournir les installations existantes en énergie. Afin d’éviter la perturbation des réseaux de transmission sans fil, l’usage d’une transmission vers un satellite dédié permettrait d’éviter de bruiter les autres modes de transmission. Afin d’exploiter les réseaux déjà existants et de simplifier le mécanisme, il serait possible de créer des sous-stations qui centralisaient la distribution et éviteraient l’exposition directe des habitations aux champs électriques créés par les lasers.


\paragraph{} Le réel défi technique sera de produire des lasers émettant sur de très basses fréquences avec une grande puissance et qui adopteront une forme d’onde limitant la quantité d’harmoniques présentes. Par la suite, la synchronisation avec le satellite et les calculs entourant l’optimisation de la réflexion et les pertes de transports seront à considérer afin d’optimiser le rendement de transmission. On peut par la suite penser aux dispositifs de transformation de tension qui devront être conçus, si le signal n’est pas un sinus pur, pour convertir de la tension relativement « sale » en tension admissible pour les appareils conventionnels. Aussi, afin de limiter le temps d’exposition, il sera important de considérer l’usage de batteries dans le but d’entreposer une partie de l’énergie. La question technique suivante se pose : « Peut-on espérer un laser émettant sur seulement quelques secondes pour fournir un nombre restreint de clients en région éloignée? » 
\section{Transport électrique}
Dans le domaine du transport électrique, on peut songer à deux approches, d'une part, on imagine des sources de lasers proches des centrales de production qui utilisent un système de réflexion comme précédemment, mais on peut aussi songer, dans ce cas, à des lasers de moins grande puissance, distribués à travers le réseau. Si l'on emploie des véhicules électriques (à batterie ou à énergie solaire), on pourrait se servir de pulsations laser afin de les alimenter en route. Tout comme on fait actuellement avec la téléphonie cellulaire, il serait possible de limiter la portée des lasers en en utilisant plusieurs, répartis sur le réseau. Dans le domaine aérien, l'élimination de la nécessité de panneaux solaires et de pétrole, permettrait de diminuer substantiellement le poids des appareils. 

\paragraph{} Tout comme dans le cas de la distribution résidentielle, les défis seront dans la conception des lasers ainsi que dans les dispositifs de conversion énergétique. Afin de limiter les pertes, il sera nécessaire d'avoir des équipements embarqués avec des rendements très élevés. Aussi, si la solution la plus intéressante s'avérait être celle des sources laser distribuées, l'optimisation du choix du laser et la porté du système sera à définir. 